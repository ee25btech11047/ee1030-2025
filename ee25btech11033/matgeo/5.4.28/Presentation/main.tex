\documentclass{beamer}
\usepackage[utf8]{inputenc}

\usetheme{Madrid}
\usecolortheme{default}
\usepackage{amsmath,amssymb,amsfonts,amsthm}
\usepackage{txfonts}
\usepackage{tkz-euclide}
\usepackage{listings}
\usepackage{adjustbox}
\usepackage{array}
\usepackage{tabularx}
\usepackage{gvv}
\usepackage{lmodern}
\usepackage{circuitikz}
\usepackage{tikz}
\usepackage{graphicx}

\setbeamertemplate{page number in head/foot}[totalframenumber]

\usepackage{tcolorbox}
\tcbuselibrary{minted,breakable,xparse,skins}



\definecolor{bg}{gray}{0.95}
\DeclareTCBListing{mintedbox}{O{}m!O{}}{%
  breakable=true,
  listing engine=minted,
  listing only,
  minted language=#2,
  minted style=default,
  minted options={%
    linenos,
    gobble=0,
    breaklines=true,
    breakafter=,,
    fontsize=\small,
    numbersep=8pt,
    #1},
  boxsep=0pt,
  left skip=0pt,
  right skip=0pt,
  left=25pt,
  right=0pt,
  top=3pt,
  bottom=3pt,
  arc=5pt,
  leftrule=0pt,
  rightrule=0pt,
  bottomrule=2pt,
  toprule=2pt,
  colback=bg,
  colframe=orange!70,
  enhanced,
  overlay={%
    \begin{tcbclipinterior}
    \fill[orange!20!white] (frame.south west) rectangle ([xshift=20pt]frame.north west);
    \end{tcbclipinterior}},
  #3,
}
\lstset{
    language=C,
    basicstyle=\ttfamily\small,
    keywordstyle=\color{blue},
    stringstyle=\color{orange},
    commentstyle=\color{green!60!black},
    numbers=left,
    numberstyle=\tiny\color{gray},
    breaklines=true,
    showstringspaces=false,
}
\begin{document}

\title 
{5.4.28}
\date{September 13,2025}


\author 
{Kavin B-EE25BTECH11033}






\frame{\titlepage}
\begin{frame}{Question}
Using elementary transformations, find the inverse of the following matrix.
\begin{align*}
    \myvec{2 & 4\\-5 & -1}
\end{align*}
\bigskip
\end{frame}



\begin{frame}{Theoretical Solution}

Given the matrix,
\begin{align}
    \vec{A} = \myvec{2 & 4\\-5 & -1}
\end{align}
Let $\vec{A}^{-1}$ be the inverse of the matrix $\vec{A}$.\\
\\
We know that ,
\begin{align}
    \vec{A}\vec{A}^{-1} = \vec{I}
\end{align}
\end{frame}

\begin{frame}{Theoretical Solution}
The augmented matrix of \augvec{1}{1}{\vec{A} & \vec{I}} is given by,
\begin{align}
    \augvec{2}{2}{2 & 4 & 1 & 0\\-5 & -1 & 0 & 1}
\end{align}
\begin{align}
    R_1\rightarrow\frac{1}{2}R_1\implies\augvec{2}{2}{1 & 2 & 1/2 & 0\\-5 & -1 & 0 & 1}
\end{align}
\begin{align}
    R_2 \rightarrow R_2 + 5R_1\implies\augvec{2}{2}{1 & 2 & 1/2 & 0\\0 & 9 & 5/2 & 1}
\end{align}
\begin{align}
    R_2 \rightarrow \frac{1}{9}R_2\implies\augvec{2}{2}{1 & 2 & 1/2 & 0\\0 & 1 & 5/18 & 1/9}
\end{align}
\begin{align}
    R_1 \rightarrow R_1 - 2R_2\implies\augvec{2}{2}{1 & 0 & -1/18 & -2/9\\0 & 1 & 5/18 & 1/9}
\end{align}
\end{frame}


\begin{frame}{Theoretical Solution}
\begin{align}
    \implies \vec{A}^{-1} = \myvec{-1/18 & -2/9\\5/18 & 1/9}
\end{align}
\end{frame}


\end{document}
