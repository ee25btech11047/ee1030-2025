\documentclass{beamer}
\usepackage[utf8]{inputenc}

\usetheme{Madrid}
\usecolortheme{default}
\usepackage{amsmath,amssymb,amsfonts,amsthm}
\usepackage{txfonts}
\usepackage{tkz-euclide}
\usepackage{listings}
\usepackage{adjustbox}
\usepackage{array}
\usepackage{tabularx}
\usepackage{gvv}
\usepackage{lmodern}
\usepackage{circuitikz}
\usepackage{tikz}
\usepackage{graphicx}

\setbeamertemplate{page number in head/foot}[totalframenumber]

\usepackage{tcolorbox}
\tcbuselibrary{minted,breakable,xparse,skins}



\definecolor{bg}{gray}{0.95}
\DeclareTCBListing{mintedbox}{O{}m!O{}}{%
	breakable=true,
	listing engine=minted,
	listing only,
	minted language=#2,
	minted style=default,
	minted options={%
		linenos,
		gobble=0,
		breaklines=true,
		breakafter=,,
		fontsize=\small,
		numbersep=8pt,
		#1},
	boxsep=0pt,
	left skip=0pt,
	right skip=0pt,
	left=25pt,
	right=0pt,
	top=3pt,
	bottom=3pt,
	arc=5pt,
	leftrule=0pt,
	rightrule=0pt,
	bottomrule=2pt,
	toprule=2pt,
	colback=bg,
	colframe=orange!70,
	enhanced,
	overlay={%
		\begin{tcbclipinterior}
			\fill[orange!20!white] (frame.south west) rectangle ([xshift=20pt]frame.north west);
	\end{tcbclipinterior}},
	#3,
}
\lstset{
	language=C,
	basicstyle=\ttfamily\small,
	keywordstyle=\color{blue},
	stringstyle=\color{orange},
	commentstyle=\color{green!60!black},
	numbers=left,
	numberstyle=\tiny\color{gray},
	breaklines=true,
	showstringspaces=false,
}
%------------------------------------------------------------
%This block of code defines the information to appear in the
%Title page
\title %optional
{5.9.1}
%\subtitle{A short story}

\author % (optional)
{RAVULA SHASHANK REDDY - EE25BTECH11047}

 \begin{document}
	
	
	\frame{\titlepage}
	\begin{frame}{Question}
    A fraction becomes $\tfrac{1}{3}$ when $1$ is subtracted from its numerator and it becomes 
$\tfrac{1}{4}$ when $8$ is added to its denominator. Find the fraction.

\end{frame}
\begin{frame}{Solution}
    Let fraction be  $\tfrac{x}{y}$  
\begin{align}
\vec{x}&=\myvec{x \\ y}\\
\frac{x-1}{y}&=\frac{1}{3}\\
\myvec{3 \\ -1}^T\vec{x}&=3\\
\frac{x}{y+8}&=\frac{1}{4}\\
\myvec{4 \\ -1}^T\vec{x}&=8\\
\myvec{3 & -1 \\ 4 & -1}\,\vec{x} &= \myvec{3 \\ 8}
\end{align}
\end{frame}
\begin{frame}{Solution}
\begin{align}
\myvec{3 & -1 & 3 \\ 4 & -1 & 8}
\xrightarrow{R_2 \rightarrow{R_2-R_1}}
\myvec{3 & -1 & 3 \\ 1 & 0 & 5}
\xrightarrow{R_1\rightarrow{R_1-3R_2}}
\myvec{0 & -1 & -12 \\ 1 & 0 & 5}
\end{align}

\begin{align}
\vec{x}=\myvec{5 \\ 12}
\end{align}
\begin{center}

$\boxed{Required\quad Fraction = \frac{5}{12}}$
\end{center}

\end{frame}
\begin{frame}[fragile]
\frametitle{C Code}
\begin{lstlisting}
    #include <stdio.h>

int main() {
    // Augmented matrix [A|b]
    double mat[2][3] = {
        {3, -1, 3},   // 3x - y = 3
        {4, -1, 8}    // 4x - y = 8
    };

    // Row operation: R2 -> R2 - (4/3)R1
    double factor = mat[1][0] / mat[0][0];
    for (int j = 0; j < 3; j++) {
        mat[1][j] = mat[1][j] - factor * mat[0][j];
    }
\end{lstlisting}  
\end{frame}
\begin{frame}[fragile]
\frametitle{C Code}
\begin{lstlisting}
    // Scale R2 to make pivot = 1
    double scale = mat[1][1];
    for (int j = 0; j < 3; j++) {
        mat[1][j] /= scale;
    }

    // Back substitution
    double y = mat[1][2];
    double x = (mat[0][2] - mat[0][1] * y) / mat[0][0];

    printf("x = %.2f\n", x);
    printf("y = %.2f\n", y);
    printf("Fraction = %.2f/%.2f\n", x, y);

    return 0;
}

\end{lstlisting}  
\end{frame}
\begin{frame}[fragile]
\frametitle{Python Direct}
\begin{lstlisting}
    import numpy as np
import libs.line.funcs as line
import libs.triangle.funcs as tri

# Coefficient matrix A and RHS vector b
A = np.array([[3, -1],
              [4, -1]], dtype=float)
b = np.array([3, 8], dtype=float)

# Solve system A x = b
x = np.linalg.solve(A, b)

num, den = x[0], x[1]

print("x =", num)
print("y =", den)
print("Fraction = {}/{}".format(int(num), int(den)))

\end{lstlisting}  
\end{frame}
\begin{frame}[fragile]
\frametitle{Python Shared}
\begin{lstlisting}
    import numpy as np
import ctypes

# Load shared object
solver = ctypes.CDLL("./solver.so")

# Define return types and arguments
solver.solve.argtypes = [np.ctypeslib.ndpointer(dtype=np.float64, ndim=1, flags="C")]

# Prepare output array
out = np.zeros(2, dtype=np.float64)
\end{lstlisting}  
\end{frame}
\begin{frame}[fragile]
\frametitle{Python Shared}
\begin{lstlisting}
# Call C function
solver.solve(out)

x, y = out[0], out[1]

print("x =", x)
print("y =", y)
print("Fraction = {}/{}".format(int(x), int(y)))

\end{lstlisting}  
\end{frame}
\end{document}
