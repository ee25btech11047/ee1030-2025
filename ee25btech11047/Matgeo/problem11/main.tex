\let\negmedspace\undefined
\let\negthickspace\undefined
\documentclass[journal]{article}
\usepackage[a5paper, margin=10mm, onecolumn]{geometry}
\usepackage{lmodern} % Ensure lmodern is loaded for pdflatex

\setlength{\headheight}{1cm} % Set the height of the header box
\setlength{\headsep}{0mm}     % Set the distance between the header box and the top of the text

\usepackage{gvv-book}
\usepackage{gvv}
\usepackage{cite}
\usepackage{textcomp}
\usepackage{amsmath,amssymb,amsfonts,amsthm}
\usepackage{algorithmic}
\usepackage{graphicx}
\graphicspath{{./figs/}}
\usepackage{textcomp}
\usepackage{xcolor}
\usepackage{txfonts}
\usepackage{listings}
\usepackage{enumitem}
\usepackage{mathtools}
\usepackage{gensymb}
\usepackage{comment}
\usepackage[breaklinks=true]{hyperref}
\usepackage{tkz-euclide} 
\usepackage{listings}
\usepackage{gvv}                                        
\def\inputGnumericTable{}                                 
\usepackage[latin1]{inputenc}                                
\usepackage{color}                                            
\usepackage{array}                                            
\usepackage{longtable}                                       
\usepackage{calc}                                             
\usepackage{multirow}                                         
\usepackage{hhline}                                           
\usepackage{ifthen}                                           
\usepackage{lscape}
\usepackage{circuitikz}
\tikzstyle{block} = [rectangle, draw, fill=blue!20, 
text width=4em, text centered, rounded corners, minimum height=3em]
\tikzstyle{sum} = [draw, fill=blue!10, circle, minimum size=1cm, node distance=1.5cm]
\tikzstyle{input} = [coordinate]
\tikzstyle{output} = [coordinate]


\begin{document}
	
	\bibliographystyle{IEEEtran}
	\vspace{3cm}
	
\title{5.4.41}
\author{EE25BTECH11047 - RAVULA SHASHANK REDDY}
\maketitle
\hrulefill
\bigskip 

\renewcommand{\thetable}{\theenumi}
\setlength{\intextsep}{10pt}

\textbf{Question:} \\

Using elementary transformations, find the inverse of the matrix
\begin{align*}
    \vec{A}=\myvec{2 & 1 & 3 \\[4pt] 4 & -1 & 0 \\[4pt] -7 & 2 & 1}.
\end{align*}

\textbf{Solution:}

\begin{align}
\vec{A}.\vec{A}^{-1}&=\vec{I}\\
[\vec{A}\mid \vec{I}] &= 
\myvec{2 & 1 & 3 & 1 & 0 & 0 \\[4pt]
       4 & -1 & 0 & 0 & 1 & 0 \\[4pt]
      -7 & 2 & 1 & 0 & 0 & 1}
\end{align}

\begin{align}
&\xrightarrow{R_1 \to \tfrac12 R_1}
\myvec{1 & \tfrac12 & \tfrac32 & \tfrac12 & 0 & 0 \\
       4 & -1 & 0 & 0 & 1 & 0 \\
      -7 & 2 & 1 & 0 & 0 & 1}
\end{align}

\begin{align}
&\xrightarrow{R_2 \to R_2-4R_1,\; R_3 \to R_3+7R_1}
\myvec{1 & \tfrac12 & \tfrac32 & \tfrac12 & 0 & 0 \\
       0 & -3 & -6 & -2 & 1 & 0 \\
       0 & \tfrac{11}{2} & \tfrac{23}{2} & \tfrac{7}{2} & 0 & 1}
\end{align}

\begin{align}
&\xrightarrow{R_2 \to -\tfrac13 R_2}
\myvec{1 & \tfrac12 & \tfrac32 & \tfrac12 & 0 & 0 \\
       0 & 1 & 2 & \tfrac23 & -\tfrac13 & 0 \\
       0 & \tfrac{11}{2} & \tfrac{23}{2} & \tfrac{7}{2} & 0 & 1}
\end{align}

\begin{align}
&\xrightarrow{R_1 \to R_1-\tfrac12R_2,\; R_3 \to R_3-\tfrac{11}{2}R_2}
\myvec{1 & 0 & \tfrac12 & \tfrac16 & \tfrac16 & 0 \\
       0 & 1 & 2 & \tfrac23 & -\tfrac13 & 0 \\
       0 & 0 & \tfrac12 & -\tfrac16 & \tfrac{11}{6} & 1}
\end{align}

\begin{align}
&\xrightarrow{R_3 \to 2R_3}
\myvec{1 & 0 & \tfrac12 & \tfrac16 & \tfrac16 & 0 \\
       0 & 1 & 2 & \tfrac23 & -\tfrac13 & 0 \\
       0 & 0 & 1 & -\tfrac13 & \tfrac{11}{3} & 2}
\end{align}

\begin{align}
&\xrightarrow{R_1 \to R_1-\tfrac12R_3,\; R_2 \to R_2-2R_3}
\myvec{1 & 0 & 0 & \tfrac13 & -\tfrac53 & -1 \\
       0 & 1 & 0 & \tfrac43 & -\tfrac{23}{3} & -4 \\
       0 & 0 & 1 & -\tfrac13 & \tfrac{11}{3} & 2}
\end{align}

\begin{align}
\vec{A}^{-1} &=
\myvec{\tfrac13 & -\tfrac53 & -1 \\[6pt]
       \tfrac43 & -\tfrac{23}{3} & -4 \\[6pt]
       -\tfrac13 & \tfrac{11}{3} & 2}
\end{align}

\end{document}

\end{document}