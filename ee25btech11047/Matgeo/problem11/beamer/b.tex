\documentclass{beamer}
\usepackage[utf8]{inputenc}

\usetheme{Madrid}
\usecolortheme{default}
\usepackage{amsmath,amssymb,amsfonts,amsthm}
\usepackage{txfonts}
\usepackage{tkz-euclide}
\usepackage{listings}
\usepackage{adjustbox}
\usepackage{array}
\usepackage{tabularx}
\usepackage{gvv}
\usepackage{lmodern}
\usepackage{circuitikz}
\usepackage{tikz}
\usepackage{graphicx}

\setbeamertemplate{page number in head/foot}[totalframenumber]

\usepackage{tcolorbox}
\tcbuselibrary{minted,breakable,xparse,skins}



\definecolor{bg}{gray}{0.95}
\DeclareTCBListing{mintedbox}{O{}m!O{}}{%
	breakable=true,
	listing engine=minted,
	listing only,
	minted language=#2,
	minted style=default,
	minted options={%
		linenos,
		gobble=0,
		breaklines=true,
		breakafter=,,
		fontsize=\small,
		numbersep=8pt,
		#1},
	boxsep=0pt,
	left skip=0pt,
	right skip=0pt,
	left=25pt,
	right=0pt,
	top=3pt,
	bottom=3pt,
	arc=5pt,
	leftrule=0pt,
	rightrule=0pt,
	bottomrule=2pt,
	toprule=2pt,
	colback=bg,
	colframe=orange!70,
	enhanced,
	overlay={%
		\begin{tcbclipinterior}
			\fill[orange!20!white] (frame.south west) rectangle ([xshift=20pt]frame.north west);
	\end{tcbclipinterior}},
	#3,
}
\lstset{
	language=C,
	basicstyle=\ttfamily\small,
	keywordstyle=\color{blue},
	stringstyle=\color{orange},
	commentstyle=\color{green!60!black},
	numbers=left,
	numberstyle=\tiny\color{gray},
	breaklines=true,
	showstringspaces=false,
}
%------------------------------------------------------------
%This block of code defines the information to appear in the
%Title page
\title %optional
{5.4.41}
%\subtitle{A short story}

\author % (optional)
{RAVULA SHASHANK REDDY - EE25BTECH11047}

 \begin{document}
	
	
	\frame{\titlepage}
	\begin{frame}{Question}
    Using elementary transformations, find the inverse of the matrix
\begin{align*}
    \vec{A}=\myvec{2 & 1 & 3 \\[4pt] 4 & -1 & 0 \\[4pt] -7 & 2 & 1}.
\end{align*}

\end{frame}
\begin{frame}{Solution}

    \begin{align}
    \vec{A}.\vec{A}^{-1}&=\vec{I}\\
    [\vec{A}\mid \vec{I}] &= 
\myvec{2 & 1 & 3 & 1 & 0 & 0 \\[4pt]
       4 & -1 & 0 & 0 & 1 & 0 \\[4pt]
      -7 & 2 & 1 & 0 & 0 & 1}
\end{align}

\begin{align}
&\xrightarrow{R_1 \to \tfrac12 R_1}
\myvec{1 & \tfrac12 & \tfrac32 & \tfrac12 & 0 & 0 \\
       4 & -1 & 0 & 0 & 1 & 0 \\
      -7 & 2 & 1 & 0 & 0 & 1}
\end{align}

\begin{align}
&\xrightarrow{R_2 \to R_2-4R_1,\; R_3 \to R_3+7R_1}
\myvec{1 & \tfrac12 & \tfrac32 & \tfrac12 & 0 & 0 \\
       0 & -3 & -6 & -2 & 1 & 0 \\
       0 & \tfrac{11}{2} & \tfrac{23}{2} & \tfrac{7}{2} & 0 & 1}
\end{align}

\end{frame}
\begin{frame}{Solution}
\begin{align}
&\xrightarrow{R_2 \to -\tfrac13 R_2}
\myvec{1 & \tfrac12 & \tfrac32 & \tfrac12 & 0 & 0 \\
       0 & 1 & 2 & \tfrac23 & -\tfrac13 & 0 \\
       0 & \tfrac{11}{2} & \tfrac{23}{2} & \tfrac{7}{2} & 0 & 1}
\end{align}

\begin{align}
&\xrightarrow{R_1 \to R_1-\tfrac12R_2,\; R_3 \to R_3-\tfrac{11}{2}R_2}
\myvec{1 & 0 & \tfrac12 & \tfrac16 & \tfrac16 & 0 \\
       0 & 1 & 2 & \tfrac23 & -\tfrac13 & 0 \\
       0 & 0 & \tfrac12 & -\tfrac16 & \tfrac{11}{6} & 1}
\end{align}

\begin{align}
&\xrightarrow{R_3 \to 2R_3}
\myvec{1 & 0 & \tfrac12 & \tfrac16 & \tfrac16 & 0 \\
       0 & 1 & 2 & \tfrac23 & -\tfrac13 & 0 \\
       0 & 0 & 1 & -\tfrac13 & \tfrac{11}{3} & 2}
\end{align}
\end{frame}
\begin{frame}{Solution}
\begin{align}
&\xrightarrow{R_1 \to R_1-\tfrac12R_3,\; R_2 \to R_2-2R_3}
\myvec{1 & 0 & 0 & \tfrac13 & -\tfrac53 & -1 \\
       0 & 1 & 0 & \tfrac43 & -\tfrac{23}{3} & -4 \\
       0 & 0 & 1 & -\tfrac13 & \tfrac{11}{3} & 2}
\end{align}

\begin{align}
\vec{A}^{-1} &=
\myvec{\tfrac13 & -\tfrac53 & -1 \\[6pt]
       \tfrac43 & -\tfrac{23}{3} & -4 \\[6pt]
       -\tfrac13 & \tfrac{11}{3} & 2}
\end{align}
\end{frame}
\begin{frame}[fragile]
    \frametitle{C Code}
    \begin{lstlisting}
    #include <stdio.h>

int main() {
    int i, j, k;
    double A[3][3] = {
        {2, 1, 3},
        {4, -1, 0},
        {-7, 2, 1}
    };

    double I[3][3] = {
        {1, 0, 0},
        {0, 1, 0},
        {0, 0, 1}
    };

    // Perform Gauss-Jordan elimination
    for (i = 0; i < 3; i++) {
    \end{lstlisting}
    \end{frame}
\begin{frame}[fragile]
    \frametitle{C Code}
    \begin{lstlisting}
        // Make the diagonal element 1
        double diag = A[i][i];
        for (j = 0; j < 3; j++) {
            A[i][j] /= diag;
            I[i][j] /= diag;
        }

        // Make other elements in the column 0
        for (k = 0; k < 3; k++) {
            if (k != i) {
                double factor = A[k][i];
                for (j = 0; j < 3; j++) {
                    A[k][j] -= factor * A[i][j];
                    I[k][j] -= factor * I[i][j];
                }
            }
        }
    }
    \end{lstlisting}
    \end{frame}
\begin{frame}[fragile]
    \frametitle{C Code}
    \begin{lstlisting}

    // Print the inverse
    printf("Inverse matrix is:\n");
    for (i = 0; i < 3; i++) {
        for (j = 0; j < 3; j++) {
            printf("%8.3f ", I[i][j]);
        }
        printf("\n");
    }

    return 0;
}
\end{lstlisting}
\end{frame}
\begin{frame}[fragile]
    \frametitle{Python Direct}
    \begin{lstlisting}
    import numpy as np
import libs.line.funcs as line
import libs.triangle.funcs as triangle

# Given matrix
A = np.array([[2, 1, 3],
              [4, -1, 0],
              [-7, 2, 1]], dtype=float)

# Compute inverse using numpy
A_inv = np.linalg.inv(A)
\end{lstlisting}
\end{frame}
\begin{frame}[fragile]
    \frametitle{Python Direct}
    \begin{lstlisting}
# Display results
print("Matrix A:")
print(A)
print("\nInverse of A:")
print(A_inv)

# Verification
I_check = A @ A_inv
print("\nVerification A * A_inv = ")
print(I_check)

    \end{lstlisting}
\end{frame}
\begin{frame}[fragile]
    \frametitle{Python Shared}
    \begin{lstlisting}
    import ctypes
import numpy as np
import libs.line.funcs as line
import libs.triangle.funcs as triangle

# Load shared library
lib = ctypes.CDLL("./libinverse.so")

# Define function signature
lib.inverse.argtypes = [ctypes.POINTER(ctypes.c_double),
                        ctypes.POINTER(ctypes.c_double)]
lib.inverse.restype = None

# Input matrix
A = np.array([[2, 1, 3],
              [4, -1, 0],
              [-7, 2, 1]], dtype=np.double)
  \end{lstlisting}
\end{frame}
\begin{frame}[fragile]
    \frametitle{Python Shared}
    \begin{lstlisting}
A_inv = np.zeros((3,3), dtype=np.double)

# Call C function
lib.inverse(A.ctypes.data_as(ctypes.POINTER(ctypes.c_double)),
            A_inv.ctypes.data_as(ctypes.POINTER(ctypes.c_double)))

print("Matrix A:")
print(A)

print("\nInverse from C (via ctypes):")
print(A_inv)

# Verify
print("\nVerification A * A_inv =")
print(A @ A_inv)

    \end{lstlisting}
\end{frame}
\end{document}
