\documentclass{beamer}
\usepackage[utf8]{inputenc}

\usetheme{Madrid}
\usecolortheme{default}
\usepackage{amsmath,amssymb,amsfonts,amsthm}
\usepackage{txfonts}
\usepackage{tkz-euclide}
\usepackage{listings}
\usepackage{adjustbox}
\usepackage{array}
\usepackage{tabularx}
\usepackage{gvv}
\usepackage{lmodern}
\usepackage{circuitikz}
\usepackage{tikz}
\usepackage{graphicx}

\setbeamertemplate{page number in head/foot}[totalframenumber]

\usepackage{tcolorbox}
\tcbuselibrary{minted,breakable,xparse,skins}



\definecolor{bg}{gray}{0.95}
\DeclareTCBListing{mintedbox}{O{}m!O{}}{%
  breakable=true,
  listing engine=minted,
  listing only,
  minted language=#2,
  minted style=default,
  minted options={%
    linenos,
    gobble=0,
    breaklines=true,
    breakafter=,,
    fontsize=\small,
    numbersep=8pt,
    #1},
  boxsep=0pt,
  left skip=0pt,
  right skip=0pt,
  left=25pt,
  right=0pt,
  top=3pt,
  bottom=3pt,
  arc=5pt,
  leftrule=0pt,
  rightrule=0pt,
  bottomrule=2pt,
  toprule=2pt,
  colback=bg,
  colframe=orange!70,
  enhanced,
  overlay={%
    \begin{tcbclipinterior}
    \fill[orange!20!white] (frame.south west) rectangle ([xshift=20pt]frame.north west);
    \end{tcbclipinterior}},
  #3,
}
\lstset{
    language=C,
    basicstyle=\ttfamily\small,
    keywordstyle=\color{blue},
    stringstyle=\color{orange},
    commentstyle=\color{green!60!black},
    numbers=left,
    numberstyle=\tiny\color{gray},
    breaklines=true,
    showstringspaces=false,
}
%------------------------------------------------------------

\title
{5.5.20}
\date{September 13, 2025}
\author 
{AI25BTECH11003 - Bhavesh Gaikwad}



\begin{document}


\frame{\titlepage}
\begin{frame}{Question}
Using elementary row transformations, find the inverse of the matrix
$$\myvec{3 & 0 & -1 \\ 2 & 3 & 0 \\ 0 & 4 & 1}$$
\end{frame}


\begin{frame}[fragile]
    \frametitle{Theoretical Solution}
  Let $\vec{A} = \myvec{3 & 0 & -1 \\ 2 & 3 & 0 \\ 0 & 4 & 1}$\\

Augment the matrix $\vec{A}$ with the identity
\begin{align}
[\vec{A} \, | \, \vec{I}] =
\left(
\begin{array}{ccc|ccc}
3 & 0 & -1 & 1 & 0 & 0 \\
2 & 3 & 0 & 0 & 1 & 0 \\
0 & 4 & 1 & 0 & 0 & 1 \\
\end{array}
\right)
\end{align}

Row Transformation-1: $R_1 \rightarrow \frac{R_1}{3}$
\begin{align}
\left(
\begin{array}{ccc|ccc}
1 & 0 & -\frac{1}{3} & \frac{1}{3} & 0 & 0 \\
2 & 3 & 0 & 0 & 1 & 0 \\
0 & 4 & 1 & 0 & 0 & 1 \\
\end{array}
\right)
\end{align}

\end{frame}



\begin{frame}[fragile]
    \frametitle{Theoretical Solution}

Row Transformation-2: $R_2 \rightarrow R_2 - 2R_1$
\begin{align}
\left(
\begin{array}{ccc|ccc}
1 & 0 & -\frac{1}{3} & \frac{1}{3} & 0 & 0 \\
0 & 3 & \frac{2}{3} & -\frac{2}{3} & 1 & 0 \\
0 & 4 & 1 & 0 & 0 & 1 \\
\end{array}
\right)
\end{align}

Row Transformation-3: $R_2 \rightarrow \frac{R_2}{3}$
\begin{align}
\left(
\begin{array}{ccc|ccc}
1 & 0 & -\frac{1}{3} & \frac{1}{3} & 0 & 0 \\
0 & 1 & \frac{2}{9} & -\frac{2}{9} & \frac{1}{3} & 0 \\
0 & 4 & 1 & 0 & 0 & 1 \\
\end{array}
\right)
\end{align}

Row Transformations 4 and 5: Replace $R_1 \rightarrow R_1 + \frac{1}{3}R_2$ And $R_3 \rightarrow R_3 - 4R_2$
\begin{align}
\left(
\begin{array}{ccc|ccc}
1 & 0 & -\frac{7}{27} & \frac{7}{27} & \frac{1}{9} & 0 \\
0 & 1 & \frac{2}{9} & -\frac{2}{9} & \frac{1}{3} & 0 \\
0 & 0 & 1 & 8 & -4 & 9 \\
\end{array}
\right)
\end{align}
\end{frame}



\begin{frame}[fragile]
    \frametitle{Theoretical Solution}
Row Transformation-6: $R_3 \rightarrow 9R_3$
\begin{align}
\left(
\begin{array}{ccc|ccc}
1 & 0 & -\frac{7}{27} & \frac{7}{27} & \frac{1}{9} & 0 \\
0 & 1 & \frac{2}{9} & -\frac{2}{9} & \frac{1}{3} & 0 \\
0 & 0 & 1 & 8 & -4 & 9 \\
\end{array}
\right)
\end{align}

Row Transformations 7 and 8: $R_1 \rightarrow R_1 + \frac{7}{27}R_3 $ And $R_2 \rightarrow R_2 - \frac{2}{9}R_3 $
\begin{align}
\left(
\begin{array}{ccc|ccc}
1 & 0 & 0 & \frac{7}{3} & -\frac{25}{27} & \frac{7}{3} \\
0 & 1 & 0 & -2 & \frac{11}{9} & -2 \\
0 & 0 & 1 & 8 & -4 & 9 \\
\end{array}
\right)
\end{align}

The Inverse Matrix of $\vec{A}$:
\begin{align}
\vec{A}^{-1} = \myvec{ \dfrac{7}{3} & -\dfrac{25}{27} & \dfrac{7}{3} \\[1ex]
-2 & \dfrac{11}{9} & -2 \\[1ex]
8 & -4 & 9 }
\end{align}
\end{frame}


\end{document}