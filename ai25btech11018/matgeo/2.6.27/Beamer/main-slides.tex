\documentclass{beamer}
\usepackage[utf8]{inputenc}

\usetheme{Madrid}
\usecolortheme{default}
\usepackage{amsmath,amssymb,amsfonts,amsthm}
\usepackage{txfonts}
\usepackage{tkz-euclide}
\usepackage{listings}
\usepackage{adjustbox}
\usepackage{array}
\usepackage{tabularx}
\usepackage{gvv}
\usepackage{lmodern}
\usepackage{circuitikz}
\usepackage{tikz}
\usepackage{graphicx}

\setbeamertemplate{page number in head/foot}[totalframenumber]

\usepackage{tcolorbox}
\tcbuselibrary{minted,breakable,xparse,skins}



\definecolor{bg}{gray}{0.95}
\DeclareTCBListing{mintedbox}{O{}m!O{}}{%
  breakable=true,
  listing engine=minted,
  listing only,
  minted language=#2,
  minted style=default,
  minted options={%
    linenos,
    gobble=0,
    breaklines=true,
    breakafter=,,
    fontsize=\small,
    numbersep=8pt,
    #1},
  boxsep=0pt,
  left skip=0pt,
  right skip=0pt,
  left=25pt,
  right=0pt,
  top=3pt,
  bottom=3pt,
  arc=5pt,
  leftrule=0pt,
  rightrule=0pt,
  bottomrule=2pt,
  toprule=2pt,
  colback=bg,
  colframe=orange!70,
  enhanced,
  overlay={%
    \begin{tcbclipinterior}
    \fill[orange!20!white] (frame.south west) rectangle ([xshift=20pt]frame.north west);
    \end{tcbclipinterior}},
  #3,
}
\lstset{
    language=C,
    basicstyle=\ttfamily\small,
    keywordstyle=\color{blue},
    stringstyle=\color{orange},
    commentstyle=\color{green!60!black},
    numbers=left,
    numberstyle=\tiny\color{gray},
    breaklines=true,
    showstringspaces=false,
}
%------------------------------------------------------------
%This block of code defines the information to appear in the
%Title page
\title %optional
{2.6.27}
\date{September 9,2025}
%\subtitle{A short story}

\author % (optional)
{Hemanth Reddy-AI25BTECH11018}



\begin{document}


\frame{\titlepage}
\begin{frame}{Question}
If A (-5,7),B(-4,-5),C(-1,-6) and D(4,5) are the vertices of a quadrilateral, find
 the area of quadrilateral ABCD.
\end{frame}



\begin{frame}{Theoretical Solution}
\textbf{Solution:}\\
Area of quadrilateral ABCD = The area of triangle ABC + The area of triangle ACD\\Let 
$\vec{A}$ \myvec{-5\\
7},
$\vec{B}$ \myvec{-4\\
-5},
$\vec{C}$ \myvec{-1\\
-6},
$\vec{D}$ \myvec{4\\
5}
be vectors\\
\begin{align}
    \overrightarrow{AB} =   \vec{B}  -  \vec{A}  =  \myvec{1\\
-12}
\end{align}

\begin{align}
    \overrightarrow{AC}  =   \vec{C}   -   \vec{A}   =  \myvec{4\\
-13}
\end{align}
 
\begin{align}
    \overrightarrow{AD}  =   \vec{D}   -   \vec{A}   =  \myvec{9\\
-2}
\end{align}
 




\end{frame}

\begin{frame}{Theoretical Solution}

\begin{align}
ar(ABC) &= \frac{1}{2} \, \|(\vec{B} - \vec{A}) \times (\vec{C} - \vec{A}) \|  =  17.5
\end{align}
\begin{align}
ar(ACD) &= \frac{1}{2} \, \|(\vec{C} - \vec{A}) \times (\vec{D} - \vec{A}) \|  =  54.5
\end{align}

Therefore area of quadrilateral ABCD = 17.5+54.5 = 72 sq. units



\end{frame}


\begin{frame}[fragile]
    \frametitle{C Code }
    \begin{lstlisting}

#include <stdio.h>
#include <stdlib.h>

struct Point {
    int x, y;
};

// Function to calculate the cross product (magnitude) of vectors u and v
int crossProduct(int ux, int uy, int vx, int vy) {
    return ux * vy - uy * vx;
}

// Function to calculate area of triangle given vertices p1, p2, p3 using vectors
double triangleArea(struct Point p1, struct Point p2, struct Point p3) {


    \end{lstlisting}
\end{frame}

\begin{frame}[fragile]
    \frametitle{C Code }
    \begin{lstlisting}

    int ux = p2.x - p1.x;
    int uy = p2.y - p1.y;
    int vx = p3.x - p1.x;
    int vy = p3.y - p1.y;

    int cross = crossProduct(ux, uy, vx, vy);
    return abs(cross) / 2.0;
}

int main() {
    struct Point A = {-5, 7};
    struct Point B = {-4, -5};
    struct Point C = {-1, -6};
    struct Point D = {4, 5};







    \end{lstlisting}
\end{frame}


\begin{frame}[fragile]
    \frametitle{C Code}
    \begin{lstlisting}

    // Calculate areas of triangles ABC and ACD
    double areaABC = triangleArea(A, B, C);
    double areaACD = triangleArea(A, C, D);

    // Total area of quadrilateral ABCD
    double areaABCD = areaABC + areaACD;

    printf("Area of quadrilateral ABCD = %.2f\n", areaABCD);

    return 0;
}


    \end{lstlisting}
\end{frame}







\begin{frame}[fragile]
    \frametitle{Python Code}
    \begin{lstlisting}
import matplotlib.pyplot as plt

# Coordinates of the vertices
A = (-5, 7)
B = (-4, -5)
C = (-1, -6)
D = (4, 5)

# Extract x and y coordinates for plotting, closing the shape by returning to A
x = [A[0], B[0], C[0], D[0], A[0]]
y = [A[1], B[1], C[1], D[1], A[1]]

plt.figure(figsize=(8, 8))
plt.plot(x, y, 'b-o', label='Quadrilateral ABCD')

# Fill the quadrilateral for visualization
plt.fill(x, y, 'skyblue', alpha=0.4)





    \end{lstlisting}
\end{frame}


\begin{frame}[fragile]
    \frametitle{Python Code}
    \begin{lstlisting}

# Label the vertices
for point, label in zip([A, B, C, D], ['A', 'B', 'C', 'D']):
    plt.text(point[0], point[1], label, fontsize=12, fontweight='bold',
             ha='right', color='darkblue')

plt.title('Quadrilateral ABCD')
plt.xlabel('X')
plt.ylabel('Y')
plt.grid(True)
plt.axis('equal')
plt.tight_layout()

# Save the plot as PNG file
plt.savefig('quadrilateral_ABCD.png', dpi=200)
plt.close()

print('Plot saved as quadrilateral_ABCD.png')



    \end{lstlisting}
\end{frame}




  
\begin{frame}{Plot}

\begin{figure}
    \centering
    \includegraphics[width=0.5\linewidth]{Beamer/figs/quadrilateral_ABCD.png}
    \caption{}
    \label{fig:placeholder}
\end{figure}

\end{frame}




\end{document}
