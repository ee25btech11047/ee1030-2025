\documentclass{beamer}
\usepackage[utf8]{inputenc}

\usetheme{Madrid}
\usecolortheme{default}
\usepackage{amsmath,amssymb,amsfonts,amsthm}
\usepackage{txfonts}
\usepackage{tkz-euclide}
\usepackage{listings}
\usepackage{adjustbox}
\usepackage{array}
\usepackage{tabularx}
\usepackage{gvv}
\usepackage{lmodern}
\usepackage{circuitikz}
\usepackage{tikz}
\usepackage{graphicx}

\setbeamertemplate{page number in head/foot}[totalframenumber]

\usepackage{tcolorbox}
\tcbuselibrary{minted,breakable,xparse,skins}



\definecolor{bg}{gray}{0.95}
\DeclareTCBListing{mintedbox}{O{}m!O{}}{%
  breakable=true,
  listing engine=minted,
  listing only,
  minted language=#2,
  minted style=default,
  minted options={%
    linenos,
    gobble=0,
    breaklines=true,
    breakafter=,,
    fontsize=\small,
    numbersep=8pt,
    #1},
  boxsep=0pt,
  left skip=0pt,
  right skip=0pt,
  left=25pt,
  right=0pt,
  top=3pt,
  bottom=3pt,
  arc=5pt,
  leftrule=0pt,
  rightrule=0pt,
  bottomrule=2pt,
  toprule=2pt,
  colback=bg,
  colframe=orange!70,
  enhanced,
  overlay={%
    \begin{tcbclipinterior}
    \fill[orange!20!white] (frame.south west) rectangle ([xshift=20pt]frame.north west);
    \end{tcbclipinterior}},
  #3,
}
\lstset{
    language=C,
    basicstyle=\ttfamily\small,
    keywordstyle=\color{blue},
    stringstyle=\color{orange},
    commentstyle=\color{green!60!black},
    numbers=left,
    numberstyle=\tiny\color{gray},
    breaklines=true,
    showstringspaces=false,
}
%------------------------------------------------------------
%This block of code defines the information to appear in the
%Title page
\title %optional
{2.2.10}
\date{September 8,2025}
%\subtitle{A short story}

\author % (optional)
{Hemanth Reddy-AI25BTECH11018}



\begin{document}


\frame{\titlepage}
\begin{frame}{Question}
The vectors $\vec{A} = 3\hat{i} - 2\hat{j} + 2\hat{k}$ and $\vec{B} = \hat{i} - 2\hat{k}$ are the adjancent sides of a parallelogram. \\
The acute angle between its diagonals is \underline{\hspace{2cm}}.
\end{frame}



\begin{frame}{Theoretical Solution}
\textbf{Solution:}\\


The diagonals of the parallelogram are given by\\

\begin{align}
\vec{A} + \vec{B} =
\myvec{
4 \\
-2 \\
0
}
, \vec{A} - \vec{B} =
\myvec{
2 \\
-2 \\
4
}
\end{align}

The angle $\theta$ between them satisfies
$
\cos\theta 
= \frac{\vec{d}_1 ^{T}\vec{d}_2}{\|\vec{d}_1\| \, \|\vec{d}_2\|}
= \frac{(\vec{A}+\vec{B})^{T}(\vec{A}-\vec{B})}{\|\vec{A}+\vec{B}\| \, \|\vec{A}-\vec{B}\|}
= \frac{\|\vec{A}\|^2 - \|\vec{B}\|^2}{\|\vec{A}+\vec{B}\| \, \|\vec{A}-\vec{B}\|}.
$\\
\vspace{0.3cm}
Now compute:
\begin{align}
\|\vec{A}\|^2 = 3^2 + (-2)^2 + 2^2 = 17,
\qquad
\|\vec{B}\|^2 = 1^2 + 0^2 + (-2)^2 = 5
\end{align}


\end{frame}

\begin{frame}{Theoretical Solution}
\begin{center}
\begin{align}
    \vec{A} + \vec{B} = \langle 4, -2, 0 \rangle, 
\quad \|\vec{A}+\vec{B}\| = \sqrt{20} = 2\sqrt{5},
\end{align}

\begin{align}
    \vec{A} - \vec{B} = \langle 2, -2, 4 \rangle, 
\quad \|\vec{A}-\vec{B}\| = \sqrt{24} = 2\sqrt{6}.
\end{align}

Hence
\begin{align}
    \cos\theta 
= \frac{17 - 5}{(2\sqrt{5})(2\sqrt{6})} 
= \frac{12}{4\sqrt{30}} 
= \frac{3}{\sqrt{30}}.
\end{align}



\vspace{0.3cm}
Therefore, the acute angle between the diagonals is
$
\theta = \cos^{-1}\!\left(\frac{3}{\sqrt{30}}\right) \approx 56.7^\circ.
$

\end{center}

\end{frame}

\begin{frame}[fragile]
    \frametitle{C Code }
    \begin{lstlisting}
#include <stdio.h>
#include <math.h>

// Function to compute dot product of two vectors
double dot(double v1[3], double v2[3]) {
    return v1[0]*v2[0] + v1[1]*v2[1] + v1[2]*v2[2];
}

// Function to compute magnitude of vector
double magnitude(double v[3]) {
    return sqrt(dot(v, v));
}

int main() {
    // Given vectors A and B
    double A[3] = {3, -2, 2};
    double B[3] = {1, 0, -2};
    double d1[3], d2[3];
    double cos_theta, theta;

   

    \end{lstlisting}
\end{frame}




\begin{frame}[fragile]
    \frametitle{C Code}
    \begin{lstlisting}

 // Compute diagonals
    for (int i = 0; i < 3; i++) {
        d1[i] = A[i] + B[i]; // A + B
        d2[i] = A[i] - B[i]; // A - B
    }

    // Compute cosine of angle
    cos_theta = dot(d1, d2) / (magnitude(d1) * magnitude(d2));

    // Clamp value to [-1, 1] for numerical stability
    if (cos_theta > 1.0) cos_theta = 1.0;
    if (cos_theta < -1.0) cos_theta = -1.0;

    // Angle in radians
    theta = acos(cos_theta);

  



    \end{lstlisting}
\end{frame}







\begin{frame}[fragile]
    \frametitle{C Code}
    \begin{lstlisting}
  // Convert to degrees
    theta = theta * 180.0 / M_PI;

    // Ensure acute angle
    if (theta > 90.0) {
        theta = 180.0 - theta;
    }

    printf("The acute angle between diagonals is: %.2f degrees\n", theta);

    return 0;
}


    \end{lstlisting}
\end{frame}


\begin{frame}[fragile]
    \frametitle{Python Code}
    \begin{lstlisting}

import numpy as np
import matplotlib.pyplot as plt
from mpl_toolkits.mplot3d import Axes3D

# Define vectors
A = np.array([3, -2, 2])
B = np.array([1, 0, -2])

# Diagonals of parallelogram
diag1 = A + B      # A+B
diag2 = A - B      # A−B

fig = plt.figure(figsize=(7,7))
ax = fig.add_subplot(111, projection='3d')

# Plot A
ax.quiver(0, 0, 0, A, A[1], A[asset:1], color='blue', label='A', arrow_length_ratio=0.1)

    \end{lstlisting}
\end{frame}


\begin{frame}[fragile]
    \frametitle{Python Code}
    \begin{lstlisting}

# Plot B
ax.quiver(0, 0, 0, B, B[1], B[asset:1], color='green', label='B', arrow_length_ratio=0.1)
# Plot first diagonal
ax.quiver(0, 0, 0, diag1, diag1[1], diag1[asset:1], color='red', label='A+B', arrow_length_ratio=0.1)
# Plot second diagonal
ax.quiver(0, 0, 0, diag2, diag2[1], diag2[asset:1], color='purple', label='A-B', arrow_length_ratio=0.1)

ax.set_xlabel('X')
ax.set_ylabel('Y')
ax.set_zlabel('Z')
ax.set_title('3D Vectors: Sides and Diagonals of Parallelogram')
ax.legend()
plt.tight_layout()
plt.savefig('parallelogram_diagonals.png', dpi=200)
plt.close()



    \end{lstlisting}
\end{frame}




  
\begin{frame}{Plot}
\begin{figure}
    \centering
    \includegraphics[width=0.5\linewidth]{Beamer/figs/parallelogram_diagonals.png}
    \caption{}
    \label{fig:placeholder}
\end{figure}
\end{frame}




\end{document}
