\documentclass{beamer}
\usepackage[utf8]{inputenc}
\usetheme{Madrid}
\usecolortheme{default}
\usepackage{amsmath,amssymb,amsfonts,amsthm}
\usepackage{txfonts}
\usepackage{tkz-euclide}
\usepackage{listings}
\usepackage{adjustbox}
\usepackage{array}
\usepackage{tabularx}
\usepackage{gvv}
\usepackage{lmodern}
\usepackage{circuitikz}
\usepackage{tikz}
\usepackage{graphicx}

\setbeamertemplate{page number in head/foot}[totalframenumber]

\usepackage{tcolorbox}
\tcbuselibrary{minted,breakable,xparse,skins}



\definecolor{bg}{gray}{0.95}
\DeclareTCBListing{mintedbox}{O{}m!O{}}{%
  breakable=true,
  listing engine=minted,
  listing only,
  minted language=#2,
  minted style=default,
  minted options={%
    linenos,
    gobble=0,
    breaklines=true,
    breakafter=,,
    fontsize=\small,
    numbersep=8pt,
    #1},
  boxsep=0pt,
  left skip=0pt,
  right skip=0pt,
  left=25pt,
  right=0pt,
  top=3pt,
  bottom=3pt,
  arc=5pt,
  leftrule=0pt,
  rightrule=0pt,
  bottomrule=2pt,
  toprule=2pt,
  colback=bg,
  colframe=orange!70,
  enhanced,
  overlay={%
    \begin{tcbclipinterior}
    \fill[orange!20!white] (frame.south west) rectangle ([xshift=20pt]frame.north west);
    \end{tcbclipinterior}},
  #3,
}
\lstset{
    language=C,
    basicstyle=\ttfamily\small,
    keywordstyle=\color{blue},
    stringstyle=\color{orange},
    commentstyle=\color{green!60!black},
    numbers=left,
    numberstyle=\tiny\color{gray},
    breaklines=true,
    showstringspaces=false,
}
%------------------------------------------------------------
%This block of code defines the information to appear in the
%Title page
\title %optional
{3.2.32}

%\subtitle{A short story}

\author % (optional)
{Varun-ai25btech11016}



\begin{document}


\frame{\titlepage}
\begin{frame}{Question}
Draw a right triangle ABC in which BC = 12 cm, AB = 5 cm and $\angle B$ = $90^\circ$.
\end{frame}
\begin{frame}{Solution}

\textbf{Let's consider the triangle with vertices,}
\begin{align} 
\vec{A}=\myvec{0\\5}
\vec{B}=\myvec{0\\0}
\vec{C}=\myvec{12\\0}
\end{align}

\begin{figure}[h!]
    \centering
\includegraphics[width=0.6\textwidth]{figs/3.2.32.png}
    \caption{}
    \label{fig:1}
\end{figure}
\end{frame}
\begin{frame}[fragile]
    \frametitle{C Code}
\begin{lstlisting}
#include <stdio.h>

// Function to fill the coordinates of A, B, C
// coords must be a double array of size 6
// Format: [Ax, Ay, Bx, By, Cx, Cy]
void get_triangle_coords(double *coords) {
    // Right triangle with B at (0,0), C on x-axis, A on y-axis
    coords[0] = 0.0;  // A.x
    coords[1] = 5.0;  // A.y
    coords[2] = 0.0;  // B.x
    coords[3] = 0.0;  // B.y
    coords[4] = 12.0; // C.x
    coords[5] = 0.0;  // C.y
}
\end{lstlisting}
\end{frame}
\begin{frame}[fragile]
    \frametitle{C plus Py Code}
\begin{lstlisting}
import ctypes
import matplotlib.pyplot as plt

# Load shared library
lib = ctypes.CDLL('./libtriangle.so')

# Create array for coordinates
coords = (ctypes.c_double * 6)()

# Call C function
lib.get_triangle_coords(coords)

# Extract points
A = (coords[0], coords[1])
B = (coords[2], coords[3])
C = (coords[4], coords[5])
\end{lstlisting}
\end{frame}
\begin{frame}[fragile]
    \frametitle{C plus Py Code}
\begin{lstlisting}
# Plot
x_coords = [A[0], B[0], C[0], A[0]]
y_coords = [A[1], B[1], C[1], A[1]]

plt.figure(figsize=(6,6))
plt.plot(x_coords, y_coords, 'b-', linewidth=2)
plt.fill(x_coords, y_coords, 'lightblue', alpha=0.5)

for point, name in zip([A, B, C], ['A', 'B', 'C']):
    plt.text(point[0], point[1]+0.3, name, fontsize=12, ha='center')
    plt.plot(point[0], point[1], 'ro')

plt.plot([0, 0.5, 0.5, 0], [0, 0, 0.5, 0.5], 'k-')  # right angle marker
\end{lstlisting}
\end{frame}
\begin{frame}[fragile]
    \frametitle{C plus Py Code}
\begin{lstlisting}
plt.gca().set_aspect('equal', adjustable='box')
plt.xlim(-1, 13)
plt.ylim(-1, 7)
plt.title("Right Triangle ABC (from C function)")
plt.savefig("/sdcard/matrix/ee1030-2025/ai25btech11016/matgeo/3.2.32/figs/3.2.32.png")
plt.show()
\end{lstlisting}
\end{frame}
\begin{frame}[fragile]
    \frametitle{Py Code}
\begin{lstlisting}

import matplotlib.pyplot as plt

# Coordinates of the vertices
B = (0, 0)
C = (12, 0)
A = (0, 5)

# Extract x and y coordinates
x_coords = [A[0], B[0], C[0], A[0]]
y_coords = [A[1], B[1], C[1], A[1]]

# Plot triangle
plt.figure(figsize=(6,6))
plt.plot(x_coords, y_coords, 'b-', linewidth=2)
plt.fill(x_coords, y_coords, 'lightblue', alpha=0.5)
\end{lstlisting}
\end{frame}
\begin{frame}[fragile]
    \frametitle{Py Code}
\begin{lstlisting}
# Mark vertices
for point, name in zip([A, B, C], ['A', 'B', 'C']):
    plt.text(point[0], point[1]+0.3, name, fontsize=12, ha='center')
    plt.plot(point[0], point[1], 'ro')

# Right angle marker at B
plt.plot([0, 0.5, 0.5, 0], [0, 0, 0.5, 0.5], 'k-')
\end{lstlisting}
\end{frame}
\begin{frame}[fragile]
    \frametitle{Py Code}
\begin{lstlisting}
# Axes setup
plt.gca().set_aspect('equal', adjustable='box')
plt.xlim(-1, 13)
plt.ylim(-1, 7)
plt.xlabel('x-axis (cm)')
plt.ylabel('y-axis (cm)')
plt.title('Right Triangle ABC')
plt.grid(True)
plt.savefig("/sdcard/matrix/ee1030-2025/ai25btech11016/matgeo/3.2.32/figs/3.2.32.png")
plt.show()
\end{lstlisting}
\end{frame}
\end{document}