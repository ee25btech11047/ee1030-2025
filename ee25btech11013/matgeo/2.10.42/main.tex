\let\negmedspace\undefined
\let\negthickspace\undefined
\documentclass[journal]{IEEEtran}
\usepackage[a5paper, margin=10mm, onecolumn]{geometry}
%\usepackage{lmodern} % Ensure lmodern is loaded for pdflatex
\usepackage{tfrupee} % Include tfrupee package

\setlength{\headheight}{1cm} % Set the height of the header box
\setlength{\headsep}{0mm}     % Set the distance between the header box and the top of the text

\usepackage{gvv-book}
\usepackage{gvv}
\usepackage{cite}
\usepackage{amsmath,amssymb,amsfonts,amsthm}
\usepackage{algorithmic}
\usepackage{graphicx}
\usepackage{textcomp}
\usepackage{xcolor}
\usepackage{txfonts}
\usepackage{listings}
\usepackage{enumitem}
\usepackage{mathtools}
\usepackage{gensymb}
\usepackage{comment}
\usepackage[breaklinks=true]{hyperref}
\usepackage{tkz-euclide} 
\usepackage{listings}
% \usepackage{gvv}                                        
\def\inputGnumericTable{}                                 
\usepackage[latin1]{inputenc}                                
\usepackage{color}                                            
\usepackage{array}                                            
\usepackage{longtable}                                       
\usepackage{calc}                                             
\usepackage{multirow}                                         
\usepackage{hhline}                                           
\usepackage{ifthen}                                           
\usepackage{lscape}
\usepackage{circuitikz}
\tikzstyle{block} = [rectangle, draw, fill=blue!20, 
    text width=4em, text centered, rounded corners, minimum height=3em]
\tikzstyle{sum} = [draw, fill=blue!10, circle, minimum size=1cm, node distance=1.5cm]
\tikzstyle{input} = [coordinate]
\tikzstyle{output} = [coordinate]


\begin{document}

\bibliographystyle{IEEEtran}
\vspace{3cm}

\title{2.10.42}
\author{EE25BTECH11013 - Bhargav}
\maketitle
% \newpage
% \bigskip
{\let\newpage\relax\maketitle}

\renewcommand{\thefigure}{\theenumi}
\renewcommand{\thetable}{\theenumi}
\setlength{\intextsep}{10pt} % Space between text and floats


\numberwithin{equation}{enumi}
\numberwithin{figure}{enumi}
\renewcommand{\thetable}{\theenumi}
% The following content is generated based on the provided image.

\textbf{Question}:\\
If $\vec{a}, \vec{b}$ and $\vec{c}$ are unit coplanar vectors, then the scalar triple product 

$[2\vec{a} - \vec{b} \;\quad 2\vec{b} - \vec{c} \;\quad 2\vec{c} - \vec{a}] =$


\solution \\


\begin{align}
\vec{B} = \myvec{2\vec{a} - \vec{b} & 2\vec{b} - \vec{c} & 2\vec{c} - \vec{a}} = 
\myvec{
\vec{a} & \vec{b} & \vec{c}
}
\myvec{
2 & 0 & -1 \\
-1 & 2 & 0 \\
0 & -1 & 2
}
\end{align}



\begin{align}
\therefore \det(\vec{B}) = \det(\vec{A})\cdot \det(\vec{M})
\end{align}


Since $\vec{a}, \vec{b}, \vec{c}$ are coplanar,
\begin{align}
\sbrak{\vec{a} \;\quad \vec{b} \;\quad \vec{c}} = 0.
\end{align}


\begin{align}
\Rightarrow \sbrak{2\vec{a} - \vec{b}\;\quad 2\vec{b} - \vec{c}\;\quad 2\vec{c} - \vec{a}} = \det(\vec{M}) \cdot \sbrak{\vec{a}\;\quad \vec{b}\;\quad \vec{c}} = 0.
\end{align}


This can be verified by taking an example of 3 coplanar unit vectors.\\
\begin{align}
\vec{a} = \myvec{1 \\ 0 \\ 0}
\end{align}

\begin{align}
\vec{b} = \myvec{0 \\ 1 \\ 0}
\end{align}

\begin{align}
\vec{c} = \myvec{0.6 \\ 0.8 \\ 0}
\end{align}

\begin{align}
\vec{X} = \sbrak{2\vec{a}-\vec{b} \;\quad 2\vec{b}-c \;\quad 2\vec{c}-\vec{a}} 
\end{align}

From the code, it is clear that the value of $\vec{X}$ is $0$

\end{document}