\let\negmedspace\undefined
\let\negthickspace\undefined
\documentclass[journal]{IEEEtran}
\usepackage[a4paper, margin=10mm, onecolumn]{geometry}
%\usepackage{lmodern} % Ensure lmodern is loaded for pdflatex
\usepackage{tfrupee} % Include tfrupee package

\setlength{\headheight}{1cm} % Set the height of the header box
\setlength{\headsep}{0mm}  % Set the distance between the header box and the top of the text

\usepackage{gvv-book}
\usepackage{gvv}
\usepackage{cite}
\usepackage{amsmath,amssymb,amsfonts,amsthm}
\usepackage{algorithmic}
\usepackage{graphicx}
\usepackage{float}
\usepackage{textcomp}
\usepackage{xcolor}
\usepackage{txfonts}
\usepackage{listings}
\usepackage{enumitem}
\usepackage{mathtools}
\usepackage{gensymb}
\usepackage{comment}
\usepackage[breaklinks=true]{hyperref}
\usepackage{tkz-euclide} 
\usepackage{listings}
% \usepackage{gvv}                                        
\def\inputGnumericTable{}                                 
\usepackage[latin1]{inputenc}                                
\usepackage{color}                                            
\usepackage{array}                                            
\usepackage{longtable}                                       
\usepackage{calc}                                             
\usepackage{multirow}                                         
\usepackage{hhline}                                           
\usepackage{ifthen}                                           
\usepackage{lscape}
\usepackage{tikz}
\usetikzlibrary{patterns}

\begin{document}

\bibliographystyle{IEEEtran}
\vspace{3cm}

\title{5.9.17}
\author{ee25btech11063-vejith}

\maketitle
% \maketitle
% \newpage
% \bigskip
{\let\newpage\relax\maketitle}
\renewcommand{\thefigure}{\theenumi}
\renewcommand{\thetable}{\theenumi}
\setlength{\intextsep}{10pt} % Space between text and floats
\textbf{Question}\\
If   $\begin{pmatrix}
    2a+b & a-2b\\
    5c-d & 4c+3d
\end{pmatrix}$=$\begin{pmatrix}
    4 & -3\\
    11 & 24
\end{pmatrix}$, then the value of a+b-c+2d\\ \\
\textbf{Solution}:\\
From the matrix equation the first row gives
\begin{align}
    \begin{pmatrix}
        2 & 1 & 0 & 0\\
        1 & -2 & 0 & 0
        \end{pmatrix} \myvec{a\\b\\c\\d}=\myvec{4\\-3}
    \end{align}
    From the matrix equation the second row gives
    \begin{align}
    \begin{pmatrix}
        0 & 0 & 5 & -1\\
        0 & 0 & 4 & 3
        \end{pmatrix} \myvec{a\\b\\c\\d}=\myvec{11\\24}
    \end{align}
combine (1) and (2)
\begin{align}
    \begin{pmatrix}
        2 & 1 & 0 & 0\\
        1 & -2 & 0 & 0\\
        0 & 0 & 5 & -1\\
        0 & 0 & 4 & 3
        \end{pmatrix} \myvec{a\\b\\c\\d}=\myvec{4\\-3\\11\\24}
\end{align}
Forming the augmented matrix\\
\begin{align}
    \left(\begin{array}{cccc|c}
       2 & 1 & 0 & 0 & 4\\
        1 & -2 & 0 & 0 & -3\\
        0 & 0 & 5 & -1 & 11\\
        0 & 0 & 4 & 3 & 24
\end{array}\right) &\xrightarrow{R_2 \leftrightarrow R_2-\frac{1}{2}\times R_1}
\left(\begin{array}{cccc|c}
       2 & 1 & 0 & 0 & 4\\
        0 & -\frac{5}{2} & 0 & 0 & -5\\
        0 & 0 & 5 & -1 & 11\\
        0 & 0 & 4 & 3 & 24
        \end{array}\right)\\ &\xrightarrow{R_1 \leftrightarrow R_1+\frac{2}{5}\times R_2} \left(\begin{array}{cccc|c}
       2 & 0 & 0 & 0 & 2\\
        0 & -\frac{5}{2} & 0 & 0 & -5\\
        0 & 0 & 5 & -1 & 11\\
        0 & 0 & 4 & 3 & 24
        \end{array}\right)\\ &\xrightarrow{R_4 \leftrightarrow R_4-\frac{4}{5}\times R_3} \left(\begin{array}{cccc|c}
       2 & 0 & 0 & 0 & 2\\
        0 & -\frac{5}{2} & 0 & 0 & -5\\
        0 & 0 & 5 & -1 & 11\\
        0 & 0 & 0 & \frac{19}{5} & \frac{76}{5}
        \end{array}\right)\\
         &\xrightarrow{R_3 \leftrightarrow R_3+\frac{5}{19}\times R_4} \left(\begin{array}{cccc|c}
       2 & 0 & 0 & 0 & 2\\
        0 & -\frac{5}{2} & 0 & 0 & -5\\
        0 & 0 & 5 & 0 & 15\\
        0 & 0 & 0 & \frac{19}{5} & \frac{76}{5}
        \end{array}\right)
        \end{align}
        on back substitution we get
        \begin{align}
        \implies \myvec{a\\b\\c\\d}=\myvec{1\\2\\3\\4}
\end{align}
    value of 
    \begin{align}
    a+b-c+2d=    \brak{1\hspace{0.3cm}1\hspace{0.3cm}-1\hspace{0.3cm}2}\myvec{a\\b\\c\\d}\\=\brak{1\hspace{0.3cm}1\hspace{0.3cm}-1\hspace{0.3cm}2}\myvec{1\\2\\3\\4}=8
    \end{align}
\end{document}
