\documentclass{beamer}
\mode<presentation>
\usepackage{amsmath,amssymb,mathtools}
\usepackage{textcomp}
\usepackage{gensymb}
\usepackage{adjustbox}
\usepackage{subcaption}
\usepackage{enumitem}
\usepackage{multicol}
\usepackage{listings}
\usepackage{url}
\usepackage{graphicx} % <-- needed for images
\def\UrlBreaks{\do\/\do-}

\usetheme{Boadilla}
\usecolortheme{lily}
\setbeamertemplate{footline}{
  \leavevmode%
  \hbox{%
  \begin{beamercolorbox}[wd=\paperwidth,ht=2ex,dp=1ex,right]{author in head/foot}%
    \insertframenumber{} / \inserttotalframenumber\hspace*{2ex}
  \end{beamercolorbox}}%
  \vskip0pt%
}
\setbeamertemplate{navigation symbols}{}

\lstset{
  frame=single,
  breaklines=true,
  columns=fullflexible,
  basicstyle=\ttfamily\tiny   % tiny font so code fits
}

\numberwithin{equation}{section}

% ---- your macros ----
\providecommand{\nCr}[2]{\,^{#1}C_{#2}}
\providecommand{\nPr}[2]{\,^{#1}P_{#2}}
\providecommand{\mbf}{\mathbf}
\providecommand{\pr}[1]{\ensuremath{\Pr\left(#1\right)}}
\providecommand{\qfunc}[1]{\ensuremath{Q\left(#1\right)}}
\providecommand{\sbrak}[1]{\ensuremath{{}\left[#1\right]}}
\providecommand{\lsbrak}[1]{\ensuremath{{}\left[#1\right.}}
\providecommand{\rsbrak}[1]{\ensuremath{\left.#1\right]}}
\providecommand{\brak}[1]{\ensuremath{\left(#1\right)}}
\providecommand{\lbrak}[1]{\ensuremath{\left(#1\right.}}
\providecommand{\rbrak}[1]{\ensuremath{\left.#1\right)}}
\providecommand{\cbrak}[1]{\ensuremath{\left\{#1\right\}}}
\providecommand{\lcbrak}[1]{\ensuremath{\left\{#1\right.}}
\providecommand{\rcbrak}[1]{\ensuremath{\left.#1\right\}}}
\theoremstyle{remark}
\newtheorem{rem}{Remark}
\newcommand{\sgn}{\mathop{\mathrm{sgn}}}
\providecommand{\abs}[1]{\left\vert#1\right\vert}
\providecommand{\res}[1]{\Res\displaylimits_{#1}}
\providecommand{\norm}[1]{\lVert#1\rVert}
\providecommand{\mtx}[1]{\mathbf{#1}}
\providecommand{\mean}[1]{E\left[ #1 \right]}
\providecommand{\fourier}{\overset{\mathcal{F}}{ \rightleftharpoons}}
\providecommand{\system}{\overset{\mathcal{H}}{ \longleftrightarrow}}
\providecommand{\dec}[2]{\ensuremath{\overset{#1}{\underset{#2}{\gtrless}}}}
\newcommand{\myvec}[1]{\ensuremath{\begin{pmatrix}#1\end{pmatrix}}}
\let\vec\mathbf

\title{Matgeo Presentation - 5.5.15}
\author{ee25btech11063 - Vejith}

\begin{document}


\frame{\titlepage}
\begin{frame}{Question}
Using elementary row transformations,find the inverse of the matrix $\Vec{A}$=
$\begin{pmatrix}
        1 & 2 & 3 \\
        2 & 5 & 7 \\
        -2 & -4 & -5
    \end{pmatrix}$
\end{frame}

\begin{frame}{Solution}
    \begin{align}
    \Vec{A}=
\begin{pmatrix}
        1 & 2 & 3 \\
        2 & 5 & 7 \\
        -2 & -4 & -5
    \end{pmatrix}
\end{align}
The Augmented matrix is
    \begin{align}
\left(\begin{array}{c|c}
        \Vec{A} & \Vec{I}
\end{array}\right)
\implies 
\left(\begin{array}{ccc|ccc}
        1 & 2 & 3 & 1 & 0 & 0\\
        2 & 5 & 7 &  0 & 1 & 0\\
        -2 & -4 & -5 & 0 & 0 &1
\end{array}\right) \\
\xleftrightarrow{R_2 \rightarrow R_2-2R_1}  \left(\begin{array}{ccc|ccc}
        1 & 2 & 3 & 1 & 0 & 0\\
        0 & 1 & 1 &  0 & 1 & 0\\
        -2 & -4 & -5 & 0 & 0 &1
\end{array}\right)\\
\xrightarrow{R_3 \rightarrow R_3+2R_1} \left(\begin{array}{ccc|ccc}
        1 & 2 & 3 & 1 & 0 & 0\\
        0 & 1 & 1 &  -2 & 1 & 0\\
        0 & 0 & 1 & 2 & 0 &1
\end{array}\right)
\end{align}
\end{frame}

\begin{frame}{Solution}
    \begin{align}
\xrightarrow{R_1 \rightarrow R_1-3R_3} \left(\begin{array}{ccc|ccc}
        1 & 2 & 0 & -5 & 0 & 1\\
        0 & 1 & 1 &  -2 & 1 & 0\\
        0 & 0 & 1 & 2 & 0 &1
\end{array}\right)\\
\xrightarrow{R_2 \rightarrow R_2-R_3} \left(\begin{array}{ccc|ccc}
        1 & 2 & 0 & -5 & 0 & -3\\
        0 & 1 & 0 &  -4 & 1 & -1\\
        0 & 0 & 1 & 2 & 0 &1
\end{array}\right)\\
\xrightarrow{R_1 \rightarrow R_1-2R_2} \left(\begin{array}{ccc|ccc}
        1 & 0 & 0 & 3 & -2 & -1\\
        0 & 1 & 0 &  -4 & 1 & -1\\
        0 & 0 & 1 & 2 & 0 &1
\end{array}\right)
\end{align}
As the left block of the Augmented matrix is $\Vec{I}$ the right block is $\Vec{A^{-1}}$.
\begin{align}
   \implies \Vec{A^{-1}}=\begin{pmatrix}
        3 & -2 & -1\\
        -4 & 1 & -1\\
        2 & 0 &1
    \end{pmatrix}
\end{align}
\end{frame}

% --------- CODE APPENDIX ---------
\section*{Appendix: Code}

% C program
\begin{frame}[fragile]{C Code: inverse.c}
\begin{lstlisting}[language=C]
#include <stdio.h>

#define SIZE 3

int main() {
    FILE *fp;
    double A[SIZE][SIZE] = {
        {1, 2, 3},
        {2, 5, 7},
        {-2, -4, -5}
    };

    double I[SIZE][SIZE] = { {1,0,0}, {0,1,0}, {0,0,1} };

    int i, j, k;
    double factor;

    // Convert A to identity, apply same operations to I
    for (i = 0; i < SIZE; i++) {
        // Make sure pivot is not zero
        if (A[i][i] == 0.0) {
            printf("Pivot is zero, cannot continue.\n");
            return 1;
        }

        // Scale pivot row to make pivot = 1
        factor = A[i][i];
        for (j = 0; j < SIZE; j++) {
            A[i][j] /= factor;
            I[i][j] /= factor;
        }

        // Eliminate other rows
                \end{lstlisting}
\end{frame}

\begin{frame}[fragile]{C Code: inverse.c}
\begin{lstlisting}[language=C]
        for (k = 0; k < SIZE; k++) {
            if (k != i) {
                factor = A[k][i];
                for (j = 0; j < SIZE; j++) {
                    A[k][j] -= factor * A[i][j];
                    I[k][j] -= factor * I[i][j];
                }
            }
        }
    }

    // Write result to file
    fp = fopen("inverse.dat", "w");
    if (fp == NULL) {
        printf("Error opening file.\n");
        return 1;
    }

    fprintf(fp, "Inverse of the matrix is:\n");
    for (i = 0; i < SIZE; i++) {
        for (j = 0; j < SIZE; j++) {
            fprintf(fp, "%8.3f ", I[i][j]);
        }
        fprintf(fp, "\n");
    }

    fclose(fp);
    printf("Inverse written to inverse.dat\n");

    return 0;
}  \end{lstlisting}
\end{frame}

\begin{frame}[fragile]{Python: inverse.py}
\begin{lstlisting}[language=Python]
import numpy as np

# Define the matrix
A = np.array([[1, 2, 3],
              [2, 5, 7],
              [-2, -4, -5]])

# Compute inverse
A_inv = np.linalg.inv(A)

print("Inverse of the matrix is:")
print(A_inv)

# Verify by multiplying A and A_inv
identity_check = np.dot(A, A_inv)

print("\nVerification (A * A_inv):")
print(identity_check)

\end{lstlisting}
\end{frame} 
\end{document}
