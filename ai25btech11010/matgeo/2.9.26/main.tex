\let\negmedspace\undefined
\let\negthickspace\undefined
\documentclass[journal]{IEEEtran}
\usepackage[a5paper, margin=10mm, onecolumn]{geometry}
\usepackage{tfrupee}

\setlength{\headheight}{1cm}
\setlength{\headsep}{0mm}

\usepackage{gvv-book}
\usepackage{gvv}
\usepackage{cite}
\usepackage{amsmath,amssymb,amsfonts,amsthm}
\usepackage{algorithmic}
\usepackage{graphicx}
\usepackage{textcomp}
\usepackage{xcolor}
\usepackage{txfonts}
\usepackage{listings}
\usepackage{enumitem}
\usepackage{mathtools}
\usepackage{gensymb}
\usepackage{comment}
\usepackage[breaklinks=true]{hyperref}
\usepackage{tkz-euclide} 
\usepackage{listings}

\graphicspath{{./figs/}}

\begin{document}
\title{2.9.26}
\author{AI25BTECH11010 - Dhanush Kumar}
\maketitle
\renewcommand{\thefigure}{\theenumi}
\renewcommand{\thetable}{\theenumi}

\noindent
\textbf{Question}\\
If $f(\alpha) = \myvec{\cos\alpha & -\sin\alpha & 0 \\ \sin\alpha & \cos\alpha & 0 \\ 0 & 0 & 1}$,  
prove that $f(\alpha)f(-\beta) = f(\alpha-\beta)$.

\bigskip
\noindent

\textbf{Solution}\\

We have
\begin{align}
f(\theta) 
&= \myvec{\cos\theta & -\sin\theta & 0 \\ \sin\theta & \cos\theta & 0 \\ 0 & 0 & 1}, \\[1em]
f(\alpha) 
	&= \myvec{\cos\alpha & -\sin\alpha & 0 \\ \sin\alpha & \cos\alpha & 0 \\ 0 & 0 & 1}\\[1em]
f(-\beta) 
&= \myvec{\cos\beta & \sin\beta & 0 \\ -\sin\beta & \cos\beta & 0 \\ 0 & 0 & 1}, \\[1em]
f(\alpha)f(-\beta) 
&= \myvec{\cos\alpha & -\sin\alpha & 0 \\ \sin\alpha & \cos\alpha & 0 \\ 0 & 0 & 1}
   \myvec{\cos\beta & \sin\beta & 0 \\ -\sin\beta & \cos\beta & 0 \\ 0 & 0 & 1}, \\[1em]
&= \myvec{\cos\alpha\cos\beta + \sin\alpha\sin\beta & \cos\alpha\sin\beta - \sin\alpha\cos\beta & 0 \\[2mm]
          \sin\alpha\cos\beta - \cos\alpha\sin\beta & \sin\alpha\sin\beta + \cos\alpha\cos\beta & 0 \\[1mm]
          0 & 0 & 1}, \\[1em]
&= \myvec{\cos(\alpha-\beta) & -\sin(\alpha-\beta) & 0 \\ \sin(\alpha-\beta) & \cos(\alpha-\beta) & 0 \\ 0 & 0 & 1}, \\[1em]
&= f(\alpha-\beta).
\end{align}

Thus proved.

\end{document}


