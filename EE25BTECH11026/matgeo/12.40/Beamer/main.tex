\documentclass{beamer}
\usepackage[utf8]{inputenc}

\usetheme{Madrid}
\usecolortheme{default}
\usepackage{amsmath,amssymb,amsfonts,amsthm}
\usepackage{mathtools}
\usepackage{txfonts}
\usepackage{tkz-euclide}
\usepackage{listings}
\usepackage{adjustbox}
\usepackage{array}
\usepackage{gensymb}
\usepackage{tabularx}
\usepackage{gvv}
\usepackage{lmodern}
\usepackage{circuitikz}
\usepackage{tikz}
\lstset{literate={·}{{$\cdot$}}1 {λ}{{$\lambda$}}1 {→}{{$\to$}}1}
\usepackage{graphicx}

\setbeamertemplate{page number in head/foot}[totalframenumber]

\usepackage{tcolorbox}
\tcbuselibrary{minted,breakable,xparse,skins}



\definecolor{bg}{gray}{0.95}
\DeclareTCBListing{mintedbox}{O{}m!O{}}{%
  breakable=true,
  listing engine=minted,
  listing only,
  minted language=#2,
  minted style=default,
  minted options={%
    linenos,
    gobble=0,
    breaklines=true,
    breakafter=,,
    fontsize=\small,
    numbersep=8pt,
    #1},
  boxsep=0pt,
  left skip=0pt,
  right skip=0pt,
  left=25pt,
  right=0pt,
  top=3pt,
  bottom=3pt,
  arc=5pt,
  leftrule=0pt,
  rightrule=0pt,
  bottomrule=2pt,
  toprule=2pt,
  colback=bg,
  colframe=orange!70,
  enhanced,
  overlay={%
    \begin{tcbclipinterior}
    \fill[orange!20!white] (frame.south west) rectangle ([xshift=20pt]frame.north west);
    \end{tcbclipinterior}},
  #3,
}
\lstset{
    language=C,
    basicstyle=\ttfamily\small,
    keywordstyle=\color{blue},
    stringstyle=\color{orange},
    commentstyle=\color{green!60!black},
    numbers=left,
    numberstyle=\tiny\color{gray},
    breaklines=true,
    showstringspaces=false,
}
%------------------------------------------------------------
%This block of code defines the information to appear in the
%Title page
\title %optional
{12.40}
\date{September 12,2025}
%\subtitle{A short story}

\author % (optional)
{Harsha-EE25BTECH11026}



\begin{document}


\frame{\titlepage}


\begin{frame}{Question}
Given $\vec{M}=\myvec{2&&3&&7\\6&&4&&7\\4&&6&&14}$. Which of the following statements is/are correct:
\begin{enumerate}
    \item The rank of $\vec{M}$ is 2
    \item The rank of $\vec{M}$ is 3
    \item The rows of $\vec{M}$ are linearly independent 
    \item The determinant of $\vec{M}$ is zero.
\end{enumerate}
\end{frame}

\begin{frame}{Theoretical Solution}
Upon row reduction of matrix $\vec{M}$ to Row Echelon form (REF),
\begin{align}
    \myvec{2&&3&&7\\6&&4&&7\\4&&6&&14}
    \xleftrightarrow[\,R_3 \gets R_3-2 \times R_1]{\,R_2 \gets R_2 - 3 \times R_1}
    \myvec{2&&3&&7\\0&&-5&&-14\\0&&0&&0}
\end{align}
\begin{align*}
    \implies \text{$\brak{a}$ The rank of $\vec{M}$ is 2}
\end{align*}
\begin{align*}
    \text{$\brak{b}$ The determinant of $\vec{M}$ is 0}
\end{align*}

\end{frame}

\begin{frame}[fragile]
    \frametitle{C Code -Finding REF of a matrix}

    \begin{lstlisting}[language=C]
#include <stdio.h>

#define ROWS 3
#define COLS 3

void row_echelon_form(double A[ROWS][COLS]) {
    int pivot_row = 0;

    for (int pivot_col = 0; pivot_col < COLS; pivot_col++) {
        int pivot = -1;
        for (int r = pivot_row; r < ROWS; r++) {
            if (A[r][pivot_col] != 0.0) {
                pivot = r;
                break;
            }
        }
        if (pivot == -1) continue;
    \end{lstlisting}
\end{frame}

\begin{frame}[fragile]
    \frametitle{C Code -Finding REF of a matrix}

    \begin{lstlisting}[language=C]
        if (pivot != pivot_row) {
            for (int c = 0; c < COLS; c++) {
                double tmp = A[pivot_row][c];
                A[pivot_row][c] = A[pivot][c];
                A[pivot][c] = tmp;
            }
        }
        for (int r = pivot_row + 1; r < ROWS; r++) {
            if (A[r][pivot_col] != 0.0) {
                double factor = A[r][pivot_col] / A[pivot_row][pivot_col];
                for (int c = pivot_col; c < COLS; c++) {
                    A[r][c] -= factor * A[pivot_row][c];
                }
            }
        }pivot_row++;
        if (pivot_row == ROWS) break;
    }
}

    \end{lstlisting}
\end{frame}

\begin{frame}[fragile]
    \frametitle{C Code -Finding REF of a matrix}

    \begin{lstlisting}[language=C]
int matrix_rank(double A[ROWS][COLS]) {
    int rank = 0;
    for (int i = 0; i < ROWS; i++) {
        int nonzero = 0;
        for (int j = 0; j < COLS; j++) {
            if (A[i][j] != 0.0) {
                nonzero = 1;
                break;
            }
        }
        if (nonzero) rank++;
    }
    return rank;
}
    \end{lstlisting}
\end{frame}

\begin{frame}[fragile]
    \frametitle{C Code -Finding REF of a matrix}

    \begin{lstlisting}[language=C]
void solve_ref(double *out, int *rank_out) {
    double A[ROWS][COLS] = {
        {2, 3, 7},
        {6, 4, 7},
        {4, 6, 14}
    };

    row_echelon_form(A);
    *rank_out = matrix_rank(A);

    // Flatten into output buffer
    int k = 0;
    for (int i = 0; i < ROWS; i++) {
        for (int j = 0; j < COLS; j++) {
            out[k++] = A[i][j];
        }
    }
}
    \end{lstlisting}
\end{frame}



\begin{frame}[fragile]
    \frametitle{Python+C code}

    \begin{lstlisting}[language=Python]
import ctypes
import sympy as sp

lib = ctypes.CDLL("./libref_solver.so")
lib.solve_ref.argtypes = [ctypes.POINTER(ctypes.c_double), ctypes.POINTER(ctypes.c_int)]

result = (ctypes.c_double * 9)()
rank = ctypes.c_int()

# Call C function
lib.solve_ref(result, ctypes.byref(rank))

ref = sp.Matrix(3, 3, result)
print("Row Echelon Form (REF):")
sp.pprint(ref)
print("\nRank of matrix:", rank.value)
    \end{lstlisting}
\end{frame}

\begin{frame}[fragile]
    \frametitle{Python code}
    \begin{lstlisting}[language=Python]
import sympy as sp

A=sp.Matrix([[2,3,7],[6,4,7],[4,6,14]])

ref_A=A.echelon_form()
print("Row Echelon Form:")
sp.pprint(ref_A)
rank=A.rank()
print("Rank of the matrix=",rank)
    \end{lstlisting}   
\end{frame}

\end{document}